\documentclass{article}
\usepackage{kotex} % 한글 사용을 위한 패키지
\usepackage{enumitem} % 항목을 커스터마이징하기 위한 패키지
\usepackage[margin=0.5in]{geometry} % 여백 조정을 위한 패키지

\setmainhangulfont{Pretendard}

\begin{document}

% 제목
\begin{center}
    \huge\textbf{김재성}
\end{center}

\begin{center}
    010-2908-8041 · jae.seong@icloud.com · linkedin.com/in/kimjson · github.com/kimjson
\end{center}

% 섹션: 보유 기술
\begin{center}
    \section*{보유 기술}
\end{center}
\hrule
\paragraph{}
\begin{itemize}
    \setlength\itemsep{0.1em}
    \item 언어 및 프레임워크 \hspace{1em} TypeScript, React, Next.js, React Native
    \item 라이브러리 \hspace{4.1em} React Query, Recoil, React Hook Form, Tailwind CSS, Redux, RxJS
    \item 배포/빌드/테스트 \hspace{1.3em} CodePush, Webpack, Babel, Storybook, Github Action, Vercel
    \item 기타 \hspace{6.7em} Amplitude API, Figma, Sentry, Jira
\end{itemize}

% 섹션: 경력
\begin{center}
    \section*{경력}
\end{center}
\hrule
\paragraph{\newline \newline}
\textbf{퍼블리, 커리어리 팀, 시니어 소프트웨어 엔지니어} \hfill 2022.10 -- 2024.04
\begin{itemize}
    \setlength\itemsep{0.1em}
    \item React Native를 최신 안정 버전으로 업그레이드하여 미지원 버전의 React Native로 인한 버그를 해결하고, 생산성 및 실행 속도 향상.
    \item Amplitude Cohort로 추산된 비활성 사용자를 대상으로 인맥 추천 푸시 자동 전송 시스템을 구현하여 활성 사용자 수 개선.
    \item 게시물 WYSIWYG 편집기를 개발하여 컨텐츠의 가독성 향상.
    \item 디자인 토큰 및 디자인 시스템 컴포넌트를 구축하고 제품에 적용하여 팀원의 퍼블리싱 생산성과 디자이너와의 소통 비용 개선.
    \item Next.js 마이그레이션, next/image 및 트리쉐이킹 도입을 통해 페이지 로딩 속도 및 번들 크기 개선.
\end{itemize}
\paragraph{}
\textbf{뤼이드, 토익 팀, 프론트엔드 엔지니어} \hfill 2019.09 -- 2021.07
\begin{itemize}
    \setlength\itemsep{0.1em}
    \item 현대적인 웹 앱 배포 아키텍처 도입을 통한 레거시 파이프라인 개선; AWS API Gateway + Lambda + S3로 배포되어 있던 기존 아키텍처를 Vercel로 간소화해서 제품 배포 주기를 극적으로 단축
    \item 멀티 프론트엔드 아키텍처 구축; 개발 속도를 저하시키는 레거시 스택을 극복하여 랜딩 페이지를 개편할 수 있도록 Next.js의 Multi-Zone 기능을 이용해서 독립적인 스택과 배포 주기를 갖는 두 프로젝트(웹앱/랜딩)를 같은 도메인으로 배포되도록 함.
    \item React Query의 placeholderData 속성과 퍼사드 패턴을 이용해서 로딩 여부와 상관없이 동일한 플로우로 그려지는 데이터/스켈레톤 컴포넌트 패턴 도입
     \item 토익 파트 1-7 컨텐츠 형식 다양성, 다국어 지원, gRPC 연동 조건을 모두 충족하기 위한 Protocol Buffers 기반 리치 텍스트 컨텐츠 스키마 설계
     \item docs, xlsx 파일 형식으로 작성된 기존 토익 컨텐츠를 새 컨텐츠 스키마로 정규화하기 위해 XPath 기반의 XML 파서, 익스트랙터 개발
     \item 레거시 시스템의 기술 스택 현대화; 과도한 오퍼레이터 사용으로 유지보수가 불가능했던 redux-observable 에픽을 async-await으로 재작성하여 코드를 간소화하고 RxJS 메이저 버전을 7로 업그레이드.
\end{itemize}
\paragraph{}
\textbf{채널코퍼레이션, 워크인사이트 팀, 소프트웨어 엔지니어} \hfill 2018.06 -- 2019.09
\begin{itemize}
     \setlength\itemsep{0.1em}
     \item 매장 센서 가용성 모니터링 서버와 연동하여 센서 기술 지원 대시보드(이슈 트래커)를 개발함으로써 워크인사이트 기술 지원 팀의 생산성 향상
     \item 유지보수 용이를 위해 중복되는 로직 추상화; Redux 미들웨어를 이용하여 비즈니스 로직으로부터 페이지네이션 레이어 분리.
     \item 기술 스택 현대화; 리액트 라우터 메이져 버전 4로 업그레이드하면서 사라진 nested route 기능을 자체적으로 구현.
\end{itemize}
\paragraph{}
\textbf{에이알모드커뮤니케이션, 소프트웨어 엔지니어} \hfill 2017.11 -- 2018.06
\begin{itemize}
    \setlength\itemsep{0.1em}
     \item Meteor 기반으로 소방 교육 VR 게임과 연동되는 소방 교육 LMS 프로토타입 풀스택 개발
\end{itemize}
\paragraph{}
\textbf{3i Inc., 백엔드 엔지니어} \hfill 2017.07 -- 2017.11
\begin{itemize}
    \setlength\itemsep{0.1em}
     \item 아임포트 API, 배치 작업을 이용하여 일회성 결제 및 정기결제 기능 구현
\end{itemize}

% 섹션: 교육
\begin{center}
    \section*{교육}
\end{center}
\hrule
\paragraph{\newline \newline}
\textbf{한국과학기술원(KAIST) 전산학(Computer Science) 학사} \hfill 2014.02 -- 2022.08
\begin{itemize}
    \setlength\itemsep{0.1em}
    \item 운영체제 및 실험(A-), 인공지능개론(A0), 컴퓨터 네트워크(A-), 인간-컴퓨터 상호작용(A-), 전산학특강<3차원 데이터를 위한 기계학습>(A-) 등 수강.
    \item 산업기능요원 복무를 위해 8학기 휴학.
\end{itemize}

% 섹션: 프로젝트
\begin{center}
    \section*{프로젝트}
\end{center}
\hrule
\paragraph{\newline \newline}
\textbf{Pixel2Mesh 논문 재현} \hfill github.com/kimjson/pixel2mesh
\begin{itemize}
    \setlength\itemsep{0.1em}
    \item KAIST 전산학특강<3차원 데이터를 위한 기계학습> 수업 프로젝트 (본인 포함 2인)
    \item VGG16, GCN 기반의 Mesh Deformation 딥러닝 모델을 PyTorch로 구현
    \item 대부분의 설계 및 구현을 맡음. 외국인 팀 멤버와 영어로 소통하며 작동 원리를 설명하여 같이 기여할 수 있도록 함
\end{itemize}
\paragraph{}
\textbf{워치쿡} \hfill github.com/kimjson/watchCook
\begin{itemize}
    \setlength\itemsep{0.1em}
    \item 레시피의 각 과정을 단계별로 넘겨볼 수 있도록 만든 1인 개발 아이폰/애플워치 애플리케이션
    \item SwiftUI 프레임워크 기반으로 개발하고 CoreData, CloudKit를 활용하여 동기화 구현.
\end{itemize}
\paragraph{}
\textbf{js-type-writer} \hfill github.com/jiggum/js-type-writer
\begin{itemize}
    \setlength\itemsep{0.1em}
    \item KAIST 인공지능 기반 소프트웨어 공학 수업 프로젝트 (본인 포함 4인)
    \item 유전 알고리즘을 활용한 자바스크립트 자동 타입 어노테이션 휴리스틱 알고리즘
    \item TypeScript 컴파일러 API를 사용한 정적 추론을 통한 정확도 향상을 맡음.
\end{itemize}
\paragraph{}
\textbf{TaxoClass 논문 재현} \hfill github.com/team-corefinder/TaxoReplica
\begin{itemize}
    \setlength\itemsep{0.1em}
    \item KAIST 인공지능개론 수업 프로젝트 (본인 포함 4인)
    \item 라벨링 없이 문서 데이터와 카테고리 계층 구조 만으로 문서의 카테고리를 분류하는 semi-supervised 딥러닝 모델 구현 논문의 재현
    \item 계층 구조와 사전 학습 모델을 이용한 pseudo-label 생성, 자기 학습 로직 구현을 맡음.
\end{itemize}
\paragraph{}
\textbf{open-genius-lyric} \hfill github.com/kimjson/open-genius-lyric
\begin{itemize}
    \setlength\itemsep{0.1em}
    \item 스포티파이에서 재생 중인 노래의 제목을 html에서 추출, genius.com의 쿼리 URL로 검색해주는 크롬 플러그인 1인 개발.
\end{itemize}
\paragraph{}
\textbf{엄복동 봇} \hfill github.com/kimjson/ubd-bot
\begin{itemize}
    \setlength\itemsep{0.1em}
    \item 영화 제목을 멘션으로 받아 총 동원 관객 수를 UBD(자전차왕 엄복동의 총 동원 관객 수인 17만명을 1UBD로 함)로 환산해서 회신해주는 트위터 봇 1인 개발.
\end{itemize}

\end{document}
